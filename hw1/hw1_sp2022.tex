% Options for packages loaded elsewhere
\PassOptionsToPackage{unicode}{hyperref}
\PassOptionsToPackage{hyphens}{url}
\PassOptionsToPackage{dvipsnames,svgnames*,x11names*}{xcolor}
%
\documentclass[
]{article}
\usepackage{amsmath,amssymb}
\usepackage{lmodern}
\usepackage{ifxetex,ifluatex}
\ifnum 0\ifxetex 1\fi\ifluatex 1\fi=0 % if pdftex
  \usepackage[T1]{fontenc}
  \usepackage[utf8]{inputenc}
  \usepackage{textcomp} % provide euro and other symbols
\else % if luatex or xetex
  \usepackage{unicode-math}
  \defaultfontfeatures{Scale=MatchLowercase}
  \defaultfontfeatures[\rmfamily]{Ligatures=TeX,Scale=1}
\fi
% Use upquote if available, for straight quotes in verbatim environments
\IfFileExists{upquote.sty}{\usepackage{upquote}}{}
\IfFileExists{microtype.sty}{% use microtype if available
  \usepackage[]{microtype}
  \UseMicrotypeSet[protrusion]{basicmath} % disable protrusion for tt fonts
}{}
\makeatletter
\@ifundefined{KOMAClassName}{% if non-KOMA class
  \IfFileExists{parskip.sty}{%
    \usepackage{parskip}
  }{% else
    \setlength{\parindent}{0pt}
    \setlength{\parskip}{6pt plus 2pt minus 1pt}}
}{% if KOMA class
  \KOMAoptions{parskip=half}}
\makeatother
\usepackage{xcolor}
\IfFileExists{xurl.sty}{\usepackage{xurl}}{} % add URL line breaks if available
\IfFileExists{bookmark.sty}{\usepackage{bookmark}}{\usepackage{hyperref}}
\hypersetup{
  pdftitle={Modern Data Mining, HW 1},
  pdfauthor={Group Member 1; Group Member 2; Group Member 3},
  colorlinks=true,
  linkcolor=Maroon,
  filecolor=Maroon,
  citecolor=Blue,
  urlcolor=blue,
  pdfcreator={LaTeX via pandoc}}
\urlstyle{same} % disable monospaced font for URLs
\usepackage[margin=1in]{geometry}
\usepackage{graphicx}
\makeatletter
\def\maxwidth{\ifdim\Gin@nat@width>\linewidth\linewidth\else\Gin@nat@width\fi}
\def\maxheight{\ifdim\Gin@nat@height>\textheight\textheight\else\Gin@nat@height\fi}
\makeatother
% Scale images if necessary, so that they will not overflow the page
% margins by default, and it is still possible to overwrite the defaults
% using explicit options in \includegraphics[width, height, ...]{}
\setkeys{Gin}{width=\maxwidth,height=\maxheight,keepaspectratio}
% Set default figure placement to htbp
\makeatletter
\def\fps@figure{htbp}
\makeatother
\setlength{\emergencystretch}{3em} % prevent overfull lines
\providecommand{\tightlist}{%
  \setlength{\itemsep}{0pt}\setlength{\parskip}{0pt}}
\setcounter{secnumdepth}{5}
\ifluatex
  \usepackage{selnolig}  % disable illegal ligatures
\fi

\title{Modern Data Mining, HW 1}
\author{Group Member 1 \and Group Member 2 \and Group Member 3}
\date{Due: 11:59PM, Jan.~30th, 2021}

\begin{document}
\maketitle

{
\hypersetup{linkcolor=}
\setcounter{tocdepth}{4}
\tableofcontents
}
\pagebreak

\hypertarget{overview}{%
\section{Overview}\label{overview}}

This is a fast-paced course that covers a lot of material. There will be
a large amount of references. You may need to do your own research to
fill in the gaps in between lectures and homework/projects. It is
impossible to learn data science without getting your hands dirty.
Please budget your time evenly. Last-minute work ethic will not work for
this course.

Homework in this course is different from your usual homework assignment
as a typical student. Most of the time, they are built over real case
studies. While you will be applying methods covered in lectures, you
will also find that extra teaching materials appear here. The focus will
be always on the goals of the study, the usefulness of the data
gathered, and the limitations in any conclusions you may draw. Always
try to challenge your data analysis in a critical way. Frequently, there
are no unique solutions.

Case studies in each homework can be listed as your data science
projects (e.g.~on your CV) where you see fit.

\hypertarget{objectives}{%
\subsection{Objectives}\label{objectives}}

\begin{itemize}
\tightlist
\item
  Get familiar with \texttt{R-studio} and \texttt{RMarkdown}
\item
  Hands-on R
\item
  Learn data science essentials

  \begin{itemize}
  \tightlist
  \item
    gather data
  \item
    clean data
  \item
    summarize data
  \item
    display data
  \item
    conclusion
  \end{itemize}
\item
  Packages

  \begin{itemize}
  \tightlist
  \item
    \texttt{dplyr}
  \item
    \texttt{ggplot}
  \end{itemize}
\end{itemize}

\hypertarget{instructions}{%
\subsection{Instructions}\label{instructions}}

\begin{itemize}
\item
  \textbf{Homework assignments can be done in a group consisting of up
  to three members}. Please find your group members as soon as possible
  and register your group on our Canvas site.
\item
  \textbf{All work submitted should be completed in the R Markdown
  format.} You can find a cheat sheet for R Markdown
  \href{https://github.com/rstudio/cheatsheets/raw/master/rmarkdown-2.0.pdf}{here}.
  For those who have never used it before, we urge you to start this
  homework as soon as possible.
\item
  \textbf{Submit the following files, one submission for each group:}
  (1) Rmd file, (2) a compiled PDF or HTML version, and (3) all
  necessary data files if different from our source data. You may
  directly edit this .rmd file to add your answers. If you intend to
  work on the problems separately within your group, compile your
  answers into one Rmd file before submitting. We encourage that you at
  least attempt each problem by yourself before working with your
  teammates. Additionally, ensure that you can `knit' or compile your
  Rmd file. It is also likely that you need to configure Rstudio to
  properly convert files to PDF.
  \href{http://kbroman.org/knitr_knutshell/pages/latex.html\#converting-knitrlatex-to-pdf}{\textbf{These
  instructions}} might be helpful.
\item
  In general, be as concise as possible while giving a fully complete
  answer to each question. All necessary datasets are available in this
  homework folder on Canvas. Make sure to document your code with
  comments (written on separate lines in a code chunk using a hashtag
  \texttt{\#} before the comment) so the teaching fellows can follow
  along. R Markdown is particularly useful because it follows a `stream
  of consciousness' approach: as you write code in a code chunk, make
  sure to explain what you are doing outside of the chunk.
\item
  A few good or solicited submissions will be used as sample solutions.
  When those are released, make sure to compare your answers and
  understand the solutions.
\end{itemize}

\hypertarget{review-materials}{%
\subsection{Review materials}\label{review-materials}}

\begin{itemize}
\tightlist
\item
  Study Advanced R Tutorial (to include \texttt{dplyr} and
  \texttt{ggplot})
\item
  Study lecture 1: Data Acquisition and EDA
\end{itemize}

\hypertarget{case-study-1-audience-size}{%
\section{Case study 1: Audience Size}\label{case-study-1-audience-size}}

How successful is the Wharton Talk Show
\href{https://businessradio.wharton.upenn.edu/}{Business Radio Powered
by the Wharton School}

\textbf{Background:} Have you ever listened to
\href{https://www.siriusxm.com/}{SiriusXM}? Do you know there is a
\textbf{Talk Show} run by Wharton professors in Sirius Radio? Wharton
launched a talk show called
\href{https://businessradio.wharton.upenn.edu/}{Business Radio Powered
by the Wharton School} through the Sirius Radio station in January of
2014. Within a short period of time the general reaction seemed to be
overwhelmingly positive. To find out the audience size for the show, we
designed a survey and collected a data set via MTURK in May of 2014. Our
goal was to \textbf{estimate the audience size}. There were 51.6 million
Sirius Radio listeners then. One approach is to estimate the proportion
of the Wharton listeners to that of the Sirius listeners, \(p\), so that
we will come up with an audience size estimate of approximately 51.6
million times \(p\).

To do so, we launched a survey via Amazon Mechanical Turk
(\href{https://www.mturk.com/}{MTurk}) on May 24, 2014 at an offered
price of \$0.10 for each answered survey. We set it to be run for 6 days
with a target maximum sample size of 2000 as our goal. Most of the
observations came in within the first two days. The main questions of
interest are ``Have you ever listened to Sirius Radio'' and ``Have you
ever listened to Sirius Business Radio by Wharton?''. A few demographic
features used as control variables were also collected; these include
Gender, Age and Household Income.

We requested that only people in United States answer the questions.
Each person can only fill in the questionnaire once to avoid duplicates.
Aside from these restrictions, we opened the survey to everyone in MTurk
with a hope that the sample would be more randomly chosen.

The raw data is stored as \texttt{Survey\_results\_final.csv} on Canvas.

\hypertarget{data-preparation}{%
\subsection{Data preparation}\label{data-preparation}}

\begin{enumerate}
\def\labelenumi{\roman{enumi}.}
\tightlist
\item
  We need to clean and select only the variables of interest.
\end{enumerate}

Select only the variables Age, Gender, Education Level, Household Income
in 2013, Sirius Listener?, Wharton Listener? and Time used to finish the
survey.

Change the variable names to be ``age'', ``gender'', ``education'',
``income'', ``sirius'', ``wharton'', ``worktime''.

\begin{enumerate}
\def\labelenumi{\roman{enumi}.}
\setcounter{enumi}{1}
\tightlist
\item
  Handle missing/wrongly filled values of the selected variables
\end{enumerate}

As in real world data with user input, the data is incomplete, with
missing values, and has incorrect responses. There is no general rule
for dealing with these problems beyond ``use common sense.'' In whatever
case, explain what the problems were and how you addressed them. Be sure
to explain your rationale for your chosen methods of handling issues
with the data. Do not use Excel for this, however tempting it might be.

Tip: Reflect on the reasons for which data could be wrong or missing.
How would you address each case? For this homework, if you are trying to
predict missing values with regression, you are definitely overthinking.
Keep it simple.

\begin{verbatim}
## `stat_bin()` using `bins = 30`. Pick better value with `binwidth`.
\end{verbatim}

\includegraphics{hw1_sp2022_files/figure-latex/unnamed-chunk-1-1.pdf}

\begin{verbatim}
## `stat_bin()` using `bins = 30`. Pick better value with `binwidth`.
\end{verbatim}

\includegraphics{hw1_sp2022_files/figure-latex/unnamed-chunk-1-2.pdf}
iii. Brief summary

Write a brief report to summarize all the variables collected. Include
both summary statistics (including sample size) and graphical displays
such as histograms or bar charts where appropriate. Comment on what you
have found from this sample. (For example - it's very interesting to
think about why would one work for a job that pays only 10cents/each
survey? Who are those survey workers? The answer may be interesting even
if it may not directly relate to our goal.)

\begin{verbatim}
## `stat_bin()` using `bins = 30`. Pick better value with `binwidth`.
\end{verbatim}

\includegraphics{hw1_sp2022_files/figure-latex/unnamed-chunk-2-1.pdf}
\includegraphics{hw1_sp2022_files/figure-latex/unnamed-chunk-2-2.pdf}
\includegraphics{hw1_sp2022_files/figure-latex/unnamed-chunk-2-3.pdf}
\includegraphics{hw1_sp2022_files/figure-latex/unnamed-chunk-2-4.pdf}
\includegraphics{hw1_sp2022_files/figure-latex/unnamed-chunk-2-5.pdf}
\includegraphics{hw1_sp2022_files/figure-latex/unnamed-chunk-2-6.pdf}

\hypertarget{sample-properties}{%
\subsection{Sample properties}\label{sample-properties}}

The population from which the sample is drawn determines where the
results of our analysis can be applied or generalized. We include some
basic demographic information for the purpose of identifying sample
bias, if any exists. Combine our data and the general population
distribution in age, gender and income to try to characterize our sample
on hand.

\begin{enumerate}
\def\labelenumi{\roman{enumi}.}
\tightlist
\item
  Does this sample appear to be a random sample from the general
  population of the USA?
\item
  Does this sample appear to be a random sample from the MTURK
  population?
\end{enumerate}

Note: You can not provide evidence by simply looking at our data here.
For example, you need to find distribution of education in our age group
in US to see if the two groups match in distribution. You may need to
gather some background information about the MTURK population to have a
slight sense if this particular sample seem to a random sample from
there\ldots{} Please do not spend too much time gathering evidence.

\hypertarget{final-estimate}{%
\subsection{Final estimate}\label{final-estimate}}

Give a final estimate of the Wharton audience size in January 2014.
Assume that the sample is a random sample of the MTURK population, and
that the proportion of Wharton listeners vs.~Sirius listeners in the
general population is the same as that in the MTURK population. Write a
brief executive summary to summarize your findings and how you came to
that conclusion.

To be specific, you should include:

\begin{enumerate}
\def\labelenumi{\arabic{enumi}.}
\tightlist
\item
  Goal of the study
\item
  Method used: data gathering, estimation methods
\item
  Findings
\item
  Limitations of the study.
\end{enumerate}

\hypertarget{estimation-methods}{%
\subsection{Estimation Methods}\label{estimation-methods}}

In the existing survey data, we have 4 independent demographic varibles
\texttt{income}, \texttt{gender}, \texttt{education} and \texttt{age}.
To study the significance of each variable on the probability of
listening Wharton Talk Show (\(P(X)\)), we performed a multinomial
logistic regression as follow

The summary indicates that the \texttt{gender} and \texttt{education}
with Graduate or professional degree have significant effects on the
outcome (\(P(X)\)).

Therefore, we proposed following model

P(X) = P(X|M \cap G)P(M \cap G) + P(X|\bar{M} \cap G)P(\bar{M} \cap G) + P(X|M \cap \bar{G})P(M \cap \bar{G}) + P(X|\bar{M} \cap \bar{G})P(\bar{M} \cap \bar{G})

Where \(X\) is listnting Wharton Talk Show on Sirius MX, \(M\) is male,
\(G\) is having Graduate or professional degree.

However \ldots\ldots{} We have to simplify the model into the following

P = P(X|M)P(M) + P(X|~M)P(~M)

\hypertarget{new-task}{%
\subsection{New task}\label{new-task}}

Now suppose you are asked to design a study to estimate the audience
size of Wharton Business Radio Show as of today: You are given a budget
of \$1000. You need to present your findings in two months.

Write a proposal for this study which includes:

\begin{enumerate}
\def\labelenumi{\arabic{enumi}.}
\tightlist
\item
  Method proposed to estimate the audience size.
\item
  What data should be collected and where it should be sourced from.
  Please fill in the google form to list your platform where surveys
  will be launched and collected
  \href{https://forms.gle/8SmjFQ1tpqr6c4sa8}{HERE}
\end{enumerate}

A good proposal will give an accurate estimation with the least amount
of money used.

\hypertarget{case-study-2-women-in-science}{%
\section{Case study 2: Women in
Science}\label{case-study-2-women-in-science}}

Are women underrepresented in science in general? How does gender relate
to the type of educational degree pursued? Does the number of higher
degrees increase over the years? In an attempt to answer these
questions, we assembled a data set (\texttt{WomenData\_06\_16.xlsx})
from
\href{https://ncses.nsf.gov/pubs/nsf19304/digest/field-of-degree-women}{NSF}
about various degrees granted in the U.S. from 2006 to 2016. It contains
the following variables: Field (Non-science-engineering
(\texttt{Non-S\&E}) and sciences (\texttt{Computer\ sciences},
\texttt{Mathematics\ and\ statistics}, etc.)), Degree (\texttt{BS},
\texttt{MS}, \texttt{PhD}), Sex (\texttt{M}, \texttt{F}), Number of
degrees granted, and Year.

Our goal is to answer the above questions only through EDA (Exploratory
Data Analyses) without formal testing. We have provided sample R-codes
in the appendix to help you if needed.

\hypertarget{data-preparation-1}{%
\subsection{Data preparation}\label{data-preparation-1}}

\begin{enumerate}
\def\labelenumi{\arabic{enumi}.}
\tightlist
\item
  Understand and clean the data
\end{enumerate}

Notice the data came in as an Excel file. We need to use the package
\texttt{readxl} and the function \texttt{read\_excel()} to read the data
\texttt{WomenData\_06\_16.xlsx} into R.

\begin{enumerate}
\def\labelenumi{\roman{enumi}.}
\item
  Read the data into R.
\item
  Clean the names of each variables. (Change variable names to
  \texttt{Field},\texttt{Degree}, \texttt{Sex}, \texttt{Year} and
  \texttt{Number} )
\item
  Set the variable natures properly.
\item
  Any missing values?
\end{enumerate}

There is no missing values.

\begin{enumerate}
\def\labelenumi{\arabic{enumi}.}
\setcounter{enumi}{1}
\tightlist
\item
  Write a summary describing the data set provided here.
\end{enumerate}

\includegraphics{hw1_sp2022_files/figure-latex/summary-1.pdf} i. How
many fields are there in this data? There are 5 fields.

\begin{enumerate}
\def\labelenumi{\roman{enumi}.}
\setcounter{enumi}{1}
\item
  What are the degree types? There are are 3 degree types
\item
  How many year's statistics are being reported here? The data are
  reported from 2006 to 2016, 11 years in total
\end{enumerate}

\hypertarget{bs-degrees-in-2015}{%
\subsection{BS degrees in 2015}\label{bs-degrees-in-2015}}

Is there evidence that more males are in science-related fields vs
\texttt{Non-S\&E}? Provide summary statistics and a plot which shows the
number of people by gender and by field. Write a brief summary to
describe your findings.

\begin{verbatim}
## `summarise()` has grouped output by 'SE'. You can override using the `.groups` argument.
\end{verbatim}

\includegraphics{hw1_sp2022_files/figure-latex/unnamed-chunk-6-1.pdf}

In 2015, there is more female compared to male within \texttt{Non\ S\&E}
fields. However, within the science-related fields, the number of female
is almost identical compared to male.

\hypertarget{eda-bringing-type-of-degree-field-and-gender-in-2015}{%
\subsection{EDA bringing type of degree, field and gender in
2015}\label{eda-bringing-type-of-degree-field-and-gender-in-2015}}

\%\% Are we doing analysis on gender effects over B.S, M.S. and Ph.D or
degrees in the field

Describe the number of people by type of degree, field, and gender. Do
you see any evidence of gender effects over different types of degrees?
Again, provide graphs to summarize your findings.

\begin{verbatim}
## `summarise()` has grouped output by 'SE', 'Sex'. You can override using the `.groups` argument.
\end{verbatim}

\includegraphics{hw1_sp2022_files/figure-latex/unnamed-chunk-7-1.pdf}

The table and the graph above show the number of people by type of
degree, field, and gender. To summarize, within science \& engineering
related fields, there are approximately similar numbers of B.S. degree
granted to female and male in 2015. However, there are more advanced
degrees (M.S. and Ph.D.) granted to male within the field. By contrast,
within non science \& engineering related fields, more degrees are
awarded to female regardless of the types of the degree.

Furthermore, from the graph, we can see gender effect on degree for both
science \& engineering related fields and non science \& engineering
related fields, especially the latter one, as stated in the previous
paragraph. \#\# EDA bring all variables

In this last portion of the EDA, we ask you to provide evidence
numerically and graphically: Do the number of degrees change by gender,
field, and time?

\%\% Question: how should we do numerically

\begin{verbatim}
## `summarise()` has grouped output by 'SE', 'Sex', 'Year'. You can override using the `.groups` argument.
\end{verbatim}

\includegraphics{hw1_sp2022_files/figure-latex/unnamed-chunk-8-1.pdf}

\hypertarget{women-in-data-science}{%
\subsection{Women in Data Science}\label{women-in-data-science}}

Finally, is there evidence showing that women are underrepresented in
data science? Data science is an interdisciplinary field of computer
science, math, and statistics. You may include year and/or degree.

\begin{verbatim}
## `summarise()` has grouped output by 'Sex', 'Year'. You can override using the `.groups` argument.
\end{verbatim}

\includegraphics{hw1_sp2022_files/figure-latex/unnamed-chunk-9-1.pdf}

The figure above shows that all types of degree over past 10 years,
female are significantly underrepresented in the data science field;
while around 1/3 of the master degrees are awarded to female, roughly
only 1/4 of the Ph.D.~and B.S. degrees are awarded to female.

\hypertarget{final-brief-report}{%
\subsection{Final brief report}\label{final-brief-report}}

Summarize your findings focusing on answering the questions regarding if
we see consistent patterns that more males pursue science-related
fields. Any concerns with the data set? How could we improve on the
study?

Based on the analysis, the female is underrepresented in science and
engineering related fields over the past 10 years. While in this study,
the number of degree awarded for female may be similar to the male at a
given year, the ratio/percentage is not.

In this data set, for each year, more degrees are awarded to the female
compare to the male. In other words, the data set has more female
samples compare to male samples. While this could be true, one
improvement is that: instead of arbitrarily selecting samples, the ratio
between the number of degrees awarded to female and male for a given
year should reflect the actual ratio.

\hypertarget{case-study-3-major-league-baseball}{%
\section{Case study 3: Major League
Baseball}\label{case-study-3-major-league-baseball}}

We would like to explore how payroll affects performance among Major
League Baseball teams. The data is prepared in two formats record
payroll, winning numbers/percentage by team from 1998 to 2014.

Here are the datasets:

-\texttt{MLPayData\_Total.csv}: wide format -\texttt{baseball.csv}: long
format

Feel free to use either dataset to address the problems.

\hypertarget{eda-relationship-between-payroll-changes-and-performance}{%
\subsection{EDA: Relationship between payroll changes and
performance}\label{eda-relationship-between-payroll-changes-and-performance}}

Payroll may relate to performance among ML Baseball teams. One possible
argument is that what affects this year's performance is not this year's
payroll, but the amount that payroll increased from last year. Let us
look into this through EDA.

Create increment in payroll

\begin{enumerate}
\def\labelenumi{\roman{enumi}.}
\item
  To describe the increment of payroll in each year there are several
  possible approaches. Take 2013 as an example:

  \begin{itemize}
  \tightlist
  \item
    option 1: diff: payroll\_2013 - payroll\_2012
  \item
    option 2: log diff: log(payroll\_2013) - log(payroll\_2012)
  \end{itemize}
\end{enumerate}

Explain why the log difference is more appropriate in this setup.

\begin{enumerate}
\def\labelenumi{\roman{enumi}.}
\setcounter{enumi}{1}
\item
  Create a new variable
  \texttt{diff\_log=log(payroll\_2013)\ -\ log(payroll\_2012)}. Hint:
  use \texttt{dplyr::lag()} function.
\item
  Create a long data table including: team, year, diff\_log, win\_pct
\end{enumerate}

\hypertarget{exploratory-questions}{%
\subsection{Exploratory questions}\label{exploratory-questions}}

\begin{enumerate}
\def\labelenumi{\roman{enumi}.}
\item
  Which five teams had highest increase in their payroll between years
  2010 and 2014, inclusive?
\item
  Between 2010 and 2014, inclusive, which team(s) ``improved'' the most?
  That is, had the biggest percentage gain in wins?
\end{enumerate}

\hypertarget{do-log-increases-in-payroll-imply-better-performance}{%
\subsection{Do log increases in payroll imply better
performance?}\label{do-log-increases-in-payroll-imply-better-performance}}

Is there evidence to support the hypothesis that higher increases in
payroll on the log scale lead to increased performance?

Pick up a few statistics, accompanied with some data visualization, to
support your answer.

\hypertarget{comparison}{%
\subsection{Comparison}\label{comparison}}

Which set of factors are better explaining performance? Yearly payroll
or yearly increase in payroll? What criterion is being used?

\end{document}
